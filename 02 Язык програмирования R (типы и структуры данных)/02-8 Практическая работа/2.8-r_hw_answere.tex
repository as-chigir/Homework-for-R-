% Options for packages loaded elsewhere
\PassOptionsToPackage{unicode}{hyperref}
\PassOptionsToPackage{hyphens}{url}
%
\documentclass[
]{article}
\usepackage{amsmath,amssymb}
\usepackage{lmodern}
\usepackage{iftex}
\ifPDFTeX
  \usepackage[T1]{fontenc}
  \usepackage[utf8]{inputenc}
  \usepackage{textcomp} % provide euro and other symbols
\else % if luatex or xetex
  \usepackage{unicode-math}
  \defaultfontfeatures{Scale=MatchLowercase}
  \defaultfontfeatures[\rmfamily]{Ligatures=TeX,Scale=1}
\fi
% Use upquote if available, for straight quotes in verbatim environments
\IfFileExists{upquote.sty}{\usepackage{upquote}}{}
\IfFileExists{microtype.sty}{% use microtype if available
  \usepackage[]{microtype}
  \UseMicrotypeSet[protrusion]{basicmath} % disable protrusion for tt fonts
}{}
\makeatletter
\@ifundefined{KOMAClassName}{% if non-KOMA class
  \IfFileExists{parskip.sty}{%
    \usepackage{parskip}
  }{% else
    \setlength{\parindent}{0pt}
    \setlength{\parskip}{6pt plus 2pt minus 1pt}}
}{% if KOMA class
  \KOMAoptions{parskip=half}}
\makeatother
\usepackage{xcolor}
\usepackage[margin=1in]{geometry}
\usepackage{color}
\usepackage{fancyvrb}
\newcommand{\VerbBar}{|}
\newcommand{\VERB}{\Verb[commandchars=\\\{\}]}
\DefineVerbatimEnvironment{Highlighting}{Verbatim}{commandchars=\\\{\}}
% Add ',fontsize=\small' for more characters per line
\usepackage{framed}
\definecolor{shadecolor}{RGB}{248,248,248}
\newenvironment{Shaded}{\begin{snugshade}}{\end{snugshade}}
\newcommand{\AlertTok}[1]{\textcolor[rgb]{0.94,0.16,0.16}{#1}}
\newcommand{\AnnotationTok}[1]{\textcolor[rgb]{0.56,0.35,0.01}{\textbf{\textit{#1}}}}
\newcommand{\AttributeTok}[1]{\textcolor[rgb]{0.77,0.63,0.00}{#1}}
\newcommand{\BaseNTok}[1]{\textcolor[rgb]{0.00,0.00,0.81}{#1}}
\newcommand{\BuiltInTok}[1]{#1}
\newcommand{\CharTok}[1]{\textcolor[rgb]{0.31,0.60,0.02}{#1}}
\newcommand{\CommentTok}[1]{\textcolor[rgb]{0.56,0.35,0.01}{\textit{#1}}}
\newcommand{\CommentVarTok}[1]{\textcolor[rgb]{0.56,0.35,0.01}{\textbf{\textit{#1}}}}
\newcommand{\ConstantTok}[1]{\textcolor[rgb]{0.00,0.00,0.00}{#1}}
\newcommand{\ControlFlowTok}[1]{\textcolor[rgb]{0.13,0.29,0.53}{\textbf{#1}}}
\newcommand{\DataTypeTok}[1]{\textcolor[rgb]{0.13,0.29,0.53}{#1}}
\newcommand{\DecValTok}[1]{\textcolor[rgb]{0.00,0.00,0.81}{#1}}
\newcommand{\DocumentationTok}[1]{\textcolor[rgb]{0.56,0.35,0.01}{\textbf{\textit{#1}}}}
\newcommand{\ErrorTok}[1]{\textcolor[rgb]{0.64,0.00,0.00}{\textbf{#1}}}
\newcommand{\ExtensionTok}[1]{#1}
\newcommand{\FloatTok}[1]{\textcolor[rgb]{0.00,0.00,0.81}{#1}}
\newcommand{\FunctionTok}[1]{\textcolor[rgb]{0.00,0.00,0.00}{#1}}
\newcommand{\ImportTok}[1]{#1}
\newcommand{\InformationTok}[1]{\textcolor[rgb]{0.56,0.35,0.01}{\textbf{\textit{#1}}}}
\newcommand{\KeywordTok}[1]{\textcolor[rgb]{0.13,0.29,0.53}{\textbf{#1}}}
\newcommand{\NormalTok}[1]{#1}
\newcommand{\OperatorTok}[1]{\textcolor[rgb]{0.81,0.36,0.00}{\textbf{#1}}}
\newcommand{\OtherTok}[1]{\textcolor[rgb]{0.56,0.35,0.01}{#1}}
\newcommand{\PreprocessorTok}[1]{\textcolor[rgb]{0.56,0.35,0.01}{\textit{#1}}}
\newcommand{\RegionMarkerTok}[1]{#1}
\newcommand{\SpecialCharTok}[1]{\textcolor[rgb]{0.00,0.00,0.00}{#1}}
\newcommand{\SpecialStringTok}[1]{\textcolor[rgb]{0.31,0.60,0.02}{#1}}
\newcommand{\StringTok}[1]{\textcolor[rgb]{0.31,0.60,0.02}{#1}}
\newcommand{\VariableTok}[1]{\textcolor[rgb]{0.00,0.00,0.00}{#1}}
\newcommand{\VerbatimStringTok}[1]{\textcolor[rgb]{0.31,0.60,0.02}{#1}}
\newcommand{\WarningTok}[1]{\textcolor[rgb]{0.56,0.35,0.01}{\textbf{\textit{#1}}}}
\usepackage{graphicx}
\makeatletter
\def\maxwidth{\ifdim\Gin@nat@width>\linewidth\linewidth\else\Gin@nat@width\fi}
\def\maxheight{\ifdim\Gin@nat@height>\textheight\textheight\else\Gin@nat@height\fi}
\makeatother
% Scale images if necessary, so that they will not overflow the page
% margins by default, and it is still possible to overwrite the defaults
% using explicit options in \includegraphics[width, height, ...]{}
\setkeys{Gin}{width=\maxwidth,height=\maxheight,keepaspectratio}
% Set default figure placement to htbp
\makeatletter
\def\fps@figure{htbp}
\makeatother
\setlength{\emergencystretch}{3em} % prevent overfull lines
\providecommand{\tightlist}{%
  \setlength{\itemsep}{0pt}\setlength{\parskip}{0pt}}
\setcounter{secnumdepth}{-\maxdimen} % remove section numbering
\usepackage[russian]{babel}
\usepackage{hyperref}
\usepackage{amsmath}
\hypersetup{colorlinks=true,urlcolor=blue}
\ifLuaTeX
  \usepackage{selnolig}  % disable illegal ligatures
\fi
\IfFileExists{bookmark.sty}{\usepackage{bookmark}}{\usepackage{hyperref}}
\IfFileExists{xurl.sty}{\usepackage{xurl}}{} % add URL line breaks if available
\urlstyle{same} % disable monospaced font for URLs
\hypersetup{
  pdftitle={Домашнее задание},
  hidelinks,
  pdfcreator={LaTeX via pandoc}}

\title{Домашнее задание}
\author{}
\date{\vspace{-2.5em}}

\begin{document}
\maketitle

\hypertarget{ux437ux430ux434ux430ux447ux430-1}{%
\subsection{Задача 1}\label{ux437ux430ux434ux430ux447ux430-1}}

Ниже приведено описание показателей, взятых из документации к
результатам опроса:

\begin{itemize}
\tightlist
\item
  \texttt{ID}: id респондента, целочисленный тип;
\item
  \texttt{GENDER}: пол, факторный тип (значения: 1 --- женский, 2 ---
  мужской);
\item
  \texttt{YEAR}: год рождения, целочисленный тип.
\end{itemize}

Даны векторы, в которых сохранены несколько случайно выбранных значений
из показателей, описанных выше:

\begin{Shaded}
\begin{Highlighting}[]
\NormalTok{ID }\OtherTok{\textless{}{-}} \DecValTok{100}\SpecialCharTok{:}\DecValTok{108}
\NormalTok{GENDER }\OtherTok{\textless{}{-}} \FunctionTok{c}\NormalTok{(}\DecValTok{1}\NormalTok{, }\DecValTok{2}\NormalTok{, }\DecValTok{2}\NormalTok{, }\DecValTok{1}\NormalTok{, }\DecValTok{2}\NormalTok{, }\DecValTok{1}\NormalTok{, }\DecValTok{1}\NormalTok{, }\DecValTok{1}\NormalTok{)}
\NormalTok{YEAR }\OtherTok{\textless{}{-}} \FunctionTok{c}\NormalTok{(}\StringTok{"1983"}\NormalTok{, }\StringTok{" 1988"}\NormalTok{, }\StringTok{"1975 "}\NormalTok{, }\StringTok{"1980"}\NormalTok{, }\StringTok{"1977  "}\NormalTok{, }\StringTok{"1992"}\NormalTok{, }\StringTok{"1994"}\NormalTok{, }\StringTok{"1983 "}\NormalTok{)}
\end{Highlighting}
\end{Shaded}

\textbf{1.1.} Проверьте, к какому типу относятся векторы (числовой,
целочисленный, логический, строковый, факторный).

\emph{Ответ:}

\begin{Shaded}
\begin{Highlighting}[]
\FunctionTok{class}\NormalTok{(ID)}
\end{Highlighting}
\end{Shaded}

\begin{verbatim}
## [1] "integer"
\end{verbatim}

\begin{Shaded}
\begin{Highlighting}[]
\FunctionTok{class}\NormalTok{(GENDER)}
\end{Highlighting}
\end{Shaded}

\begin{verbatim}
## [1] "numeric"
\end{verbatim}

\begin{Shaded}
\begin{Highlighting}[]
\FunctionTok{class}\NormalTok{(YEAR)}
\end{Highlighting}
\end{Shaded}

\begin{verbatim}
## [1] "character"
\end{verbatim}

Векторы ID, GENDER и YEAR относятся к целочисленному, числовому и
строковому типам данных соответственно

\textbf{1.2.} Если тип какого-то вектора не соответствует заявленному в
описании выше, исправьте это, сохранив изменения в самом векторе. Среди
приведённых выше векторов «неправильных» может быть несколько.

\emph{Ответ:}

\begin{Shaded}
\begin{Highlighting}[]
\NormalTok{GENDER }\OtherTok{\textless{}{-}} \FunctionTok{as.factor}\NormalTok{(GENDER)}
\NormalTok{YEAR }\OtherTok{\textless{}{-}} \FunctionTok{as.integer}\NormalTok{(YEAR)}
\FunctionTok{class}\NormalTok{(ID)}
\end{Highlighting}
\end{Shaded}

\begin{verbatim}
## [1] "integer"
\end{verbatim}

\begin{Shaded}
\begin{Highlighting}[]
\FunctionTok{class}\NormalTok{(GENDER)}
\end{Highlighting}
\end{Shaded}

\begin{verbatim}
## [1] "factor"
\end{verbatim}

\begin{Shaded}
\begin{Highlighting}[]
\FunctionTok{class}\NormalTok{(YEAR)}
\end{Highlighting}
\end{Shaded}

\begin{verbatim}
## [1] "integer"
\end{verbatim}

\newpage

\hypertarget{ux437ux430ux434ux430ux447ux430-2}{%
\subsection{Задача 2}\label{ux437ux430ux434ux430ux447ux430-2}}

В таблицах ниже приведены данные по объёму экспорта и импорта товаров в
долларах США за 2019 год для трёх стран (данные проекта
\href{https://comtrade.un.org/}{COMTRADE}).

\begin{table}[ht!]
\centering
\caption{экспорт}
\begin{tabular}{|c|c|c|}
\hline
\textbf{Партнёр 1} & \textbf{Партнёр 2} & \textbf{Экспорт (доллары США)} \\
\hline
Канада & Нидерланды & 3 905 228 446\\
\hline
Канада & США & 336 531 873 909 \\
\hline
Нидерланды & Канада & 4 862 948 109 \\
\hline
Нидерланды & США & 29 807 484 356 \\
\hline
США & Канада & 292 338 433 401 \\
\hline
США & Нидерланды & 51 225 636 600\\
\hline
\end{tabular}
\end{table}

\begin{table}[ht!]
\centering
\caption{импорт}
\begin{tabular}{|c|c|c|}
\hline
\textbf{Партнёр 1} & \textbf{Партнёр 2} & \textbf{Импорт (доллары США)} \\
\hline
Канада & Нидерланды & 3 515 239 399\\
\hline
Канада & США & 229 687 088 046 \\
\hline
Нидерланды & Канада & 2 249 551 077 \\
\hline
Нидерланды & США & 42 262 861 193 \\
\hline
США & Канада & 326 628 559 104 \\
\hline
США & Нидерланды & 30 883 263 358\\
\hline
\end{tabular}
\end{table}

\textbf{2.1.} Создайте датафреймы \texttt{goods\_export} и
\texttt{goods\_import}, которые выглядят так, как таблицы с данными
выше.

\textbf{Подсказка:} названия столбцов датафрейма добавляются с помощью
той же функции, что и у матриц.

\emph{Ответ:}

Датафрейм goods\_export

\begin{verbatim}
##     Парнер 1   Парнер 2 Экспорт (доллары США)
## 1     Канада Нидерланды            3905228446
## 2     Канада        США          336531873909
## 3 Нидерланды     Канада            4862948109
## 4 Нидерланды        США           29807484356
## 5        США     Канада          292338433401
## 6        США Нидерланды           51225636600
\end{verbatim}

Датафрейм goods\_import

\begin{verbatim}
##     Парнер 1   Парнер 2 Импорт (доллары США)
## 1     Канада Нидерланды           3515239399
## 2     Канада        США         229687088046
## 3 Нидерланды     Канада           2249551077
## 4 Нидерланды        США          42262861193
## 5        США     Канада         326628559104
## 6        США Нидерланды          30883263358
\end{verbatim}

\textbf{2.2.} На основе данных из таблиц 1 и 2 создайте матрицы
\texttt{export\_mat} и \texttt{import\_mat}, которые будут в более
компактном виде хранить информацию об экспорте и импорте. Это должны
быть квадратные матрицы (число строк равно числу столбцов), по строкам и
столбцам должны идти названия стран: Канада, Нидерланды, США.

\textbf{Пример.} Известно, что страна A экспортирует в страну B товара
на 20 000 долларов, а B экспортирует в А товара на 40 000 долларов. При
этом мы считаем, что сама страна в себя ничего не экспортирует. Создадим
матрицу для описания экспорта двух стран A и B:

\begin{table}[ht!]
\centering
\begin{tabular}{cc}
\textbf{A} & \textbf{B} \\
0 & 20000 \\
40000 & 0 \\
\end{tabular}
\end{table}

\emph{Ответ:}

Матрица export\_mat

\begin{verbatim}
##            Канада  Нидерланды          США
## [1,]            0  3905228446 336531873909
## [2,]   4862948109           0  29807484356
## [3,] 292338433401 51225636600            0
\end{verbatim}

Матрица import\_mat

\begin{verbatim}
##            Канада  Нидерланды          США
## [1,]            0  3515239399 229687088046
## [2,]   2249551077           0  42262861193
## [3,] 326628559104 30883263358            0
\end{verbatim}

\textbf{2.3.} Используя матрицы из пункта 2.2, создайте матрицу
\texttt{diff\_mat}, которая содержит разницу между экспортом и импортом
стран.

\emph{Ответ:}

\begin{verbatim}
##            Канада  Нидерланды          США
## [1,]            0   389989047 106844785863
## [2,]   2613397032           0 -12455376837
## [3,] -34290125703 20342373242            0
\end{verbatim}

\textbf{2.4.} Используя матрицы из пункта 2.2, создайте матрицы с
логарифмированными (десятичный логарифм) значениями экспорта и импорта.
Назовите матрицы по своему усмотрению.

\emph{Ответ:}

Матрица export\_log

\begin{verbatim}
##        Канада Нидерланды      США
## [1,]     -Inf   9.591646 11.52703
## [2,]  9.68690       -Inf 10.47433
## [3,] 11.46589  10.709487     -Inf
\end{verbatim}

Матрица import\_log

\begin{verbatim}
##         Канада Нидерланды      США
## [1,]      -Inf   9.545955 11.36114
## [2,]  9.352096       -Inf 10.62596
## [3,] 11.514054  10.489723     -Inf
\end{verbatim}

\textbf{2.5.} Создайте список \texttt{L\_data}, который содержит
следующие элементы:

\begin{itemize}
\tightlist
\item
  элемент \texttt{source} со значением ``COMTRADE,
  \url{https://comtrade.un.org/}'';
\item
  элемент \texttt{year} со значением ``2019'';
\item
  элемент \texttt{countries} --- вектор с названиями стран (как в
  матрицах);
\item
  элемент \texttt{export} --- датафрейм \texttt{goods\_export};
\item
  элемент \texttt{import} --- датафрейм \texttt{goods\_import};
\item
  элемент \texttt{export\_mat} --- матрица \texttt{export\_mat};
\item
  элемент \texttt{import\_mat} --- матрица \texttt{export\_mat}.
\end{itemize}

\emph{Ответ:}

\begin{verbatim}
## $source
## [1] "COMTRADE, https://comtrade.un.org/"
## 
## $year
## [1] 2019
## 
## $countries
## [1] "Канада"     "Нидерланды" "США"       
## 
## $export
##     Парнер 1   Парнер 2 Экспорт (доллары США)
## 1     Канада Нидерланды            3905228446
## 2     Канада        США          336531873909
## 3 Нидерланды     Канада            4862948109
## 4 Нидерланды        США           29807484356
## 5        США     Канада          292338433401
## 6        США Нидерланды           51225636600
## 
## $import
##     Парнер 1   Парнер 2 Импорт (доллары США)
## 1     Канада Нидерланды           3515239399
## 2     Канада        США         229687088046
## 3 Нидерланды     Канада           2249551077
## 4 Нидерланды        США          42262861193
## 5        США     Канада         326628559104
## 6        США Нидерланды          30883263358
## 
## $export_mat
##            Канада  Нидерланды          США
## [1,]            0  3905228446 336531873909
## [2,]   4862948109           0  29807484356
## [3,] 292338433401 51225636600            0
## 
## $import_mat
##            Канада  Нидерланды          США
## [1,]            0  3515239399 229687088046
## [2,]   2249551077           0  42262861193
## [3,] 326628559104 30883263358            0
\end{verbatim}

\textbf{2.6.} Используя созданный список \texttt{L\_data} и обращаясь
только к индексам элементов, выведите на экран:

\begin{itemize}
\tightlist
\item
  третью страну в векторе \texttt{countries};
\item
  объём экспорта из Нидерландов в Канаду из матрицы
  \texttt{export\_mat};
\item
  объём импорта из США в Канаду из матрицы \texttt{import\_mat}.
\end{itemize}

\emph{Ответ:}

Третья страна в векторе countries

\begin{verbatim}
## [1] "США"
\end{verbatim}

Объём экспорта из Нидерландов в Канаду из матрицы export\_mat

\begin{verbatim}
##     Канада 
## 4862948109
\end{verbatim}

Объём импорта из США в Канаду из матрицы import\_mat

\begin{verbatim}
##       Канада 
## 326628559104
\end{verbatim}

\textbf{2.7.} Добавьте в конец списка \texttt{L\_data} элемент без
названия, который содержит строку с сегодняшней датой в произвольном
формате.

\emph{Ответ:}

\begin{verbatim}
## $source
## [1] "COMTRADE, https://comtrade.un.org/"
## 
## $year
## [1] 2019
## 
## $countries
## [1] "Канада"     "Нидерланды" "США"       
## 
## $export
##     Парнер 1   Парнер 2 Экспорт (доллары США)
## 1     Канада Нидерланды            3905228446
## 2     Канада        США          336531873909
## 3 Нидерланды     Канада            4862948109
## 4 Нидерланды        США           29807484356
## 5        США     Канада          292338433401
## 6        США Нидерланды           51225636600
## 
## $import
##     Парнер 1   Парнер 2 Импорт (доллары США)
## 1     Канада Нидерланды           3515239399
## 2     Канада        США         229687088046
## 3 Нидерланды     Канада           2249551077
## 4 Нидерланды        США          42262861193
## 5        США     Канада         326628559104
## 6        США Нидерланды          30883263358
## 
## $export_mat
##            Канада  Нидерланды          США
## [1,]            0  3905228446 336531873909
## [2,]   4862948109           0  29807484356
## [3,] 292338433401 51225636600            0
## 
## $import_mat
##            Канада  Нидерланды          США
## [1,]            0  3515239399 229687088046
## [2,]   2249551077           0  42262861193
## [3,] 326628559104 30883263358            0
## 
## [[8]]
## [1] "Wed Sep 21 14:09:05 2022"
\end{verbatim}

\end{document}
